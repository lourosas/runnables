%%%%%%%%%%%%%%%%%%%%%%%%%%%%%%%%%%%%%%%%%%%%%%%%%%%%%%%%%%%%%%%%%%%%%%
\documentclass[letterpaper]{article}
\usepackage{latexsym}
\usepackage{amsbsy}
\usepackage{graphicx}
\begin{document}
\noindent
\textbf{Use Case 3:  Monitor Launch Activities}\\
\textbf{Scope:  }Launch Simulator\\
\textbf{Level:  }Flight Controller Goal, Astronaut Goal\\
\textbf{Primary Actor:  }Flight Controller, Astronaut\\
\textbf{Secondary Actor:  }Engineer/Technician\\
\textbf{Related Use Cases:  }\textit{Use Case 2:  The Flight Controller
shall Initiate Launch}\\
\textbf{Stakeholders \& Interrests:  }
\begin{itemize}
\item Flight Controller:  Wants to
\begin{itemize}
\item Monitor launch data to
\begin{itemize}
\item Continue the launch
\item Abort the launch
\end{itemize}
\end{itemize}
\item Astronaut:  Wants to 
\begin{itemize}
\item Monitor launch data to
\begin{itemize}
\item Continue the launch
\item Abort the launch
\end{itemize}
\end{itemize}
\item Engineer/Technician:  Wants to monitor launch data
of there responsibility for launch integrity assessment:  to advise
the Flight Controller
\item Administrator:  wants a successful launch
\item Politician:  wants a successful launch for policy assessment:
to determine program funding/continuation
\item Politician:  wants tecnical success for local and national pride
\end{itemize}
\textbf{Pre-Conditions:  }The Launch Intiation Activities are
complete.  See \textit{Use Case 2:  Flight Controller shall
Initiate Launch}\\
\textbf{Post-Conditions:  }The launch vehicle separation from
payload/capsule--the launch is complete\\\\
\textbf{Main Success Scenario:  }\\
\begin{tabular}{|p{5.75cm}|p{5.75cm}|}\hline
\textbf{Flight Controller/Astronaut}&\textbf{System}\\\hline
&1. Starts the Flight Time Clock\\\hline
&2. Periodically returns the Flight Time Clock to the Actors through
the duration of the launch\\\hline
&3. Periodically monitors/returns launch data to the Actors through
the duration of the launch\\\hline
4.  Monitors launch data &\\\hline
&5. Final Stage Separation from the payload/capsule, transitions out
of launch (launch is complete)\\\hline
&6. Alerts of transition\\\hline
\end{tabular}\\\\
\textbf{Alternative Flows:}
\begin{itemize}
\item[3a.]Throughout the launch, the Flight Controller has the 
ability to abort the launch.  System reflects the abort status.
\item[3b.]Throughout the launch, the (commanding) astronaut has
the ability to abort the launch.  System reflects the abort status.
\item[3c.]Throughout the launch, the System alerts the primary actors
and the responsible secondary actors of any anomalous Launch Data.
If the System assess any of the anomalous launch data poses a threat
to the launch and/or lives of the mission crew, the System advises the
primary actors to abort the launch.  Primary Actors have final
authority on launch abort.
\item[5a.]If the final stage fails to separate from the
payload/capsule, then the System alerts the issue, the System does not
transition out of launch.  The lauch is aborted
(which can be initiated via either of the primary actors).  The
System reflects the abort status.
\end{itemize}
\textbf{Special Requirements:}
\begin{itemize}
\item Both primary actors have the ability to abort the Launch. The
System cannot abort the launch.
\item The abort can be in response to launch anomalies/issues
reported by the System, or at the order of the Flight Controller or
the (Commanding) Astronaut.
\end{itemize}
\textbf{Technology \& Variations List}
\begin{itemize}
\item[3a.]Stage separations during launch are part of the Launch
Data
\item[5a.]Final stage separation is part of the Lauch Data:
once the final stage separates from the capsule/payload, the
transition out of Launch occurs
\end{itemize}
\textbf{Open Issues}
\begin{itemize}
\item What if the Flight Time Clock fails (stops working, reports
erronious time, etc...)?
\item Coming back to the Fully Automated System question when was
discussed (and continues to be discussed) in previous Use Cases.  For
the time being, this System is not fully automated.  In the case
of a fully automated System, one of the open issues is manual override
in the event of an abort.
\end{itemize}
\end{document}
%%%%%%%%%%%%%%%%%%%%%%%%%%%%%%%%%%%%%%%%%%%%%%%%%%%%%%%%%%%%%%%%%%%%%%
