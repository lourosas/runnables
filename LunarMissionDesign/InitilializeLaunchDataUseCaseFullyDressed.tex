%%%%%%%%%%%%%%%%%%%%%%%%%%%%%%%%%%%%%%%%%%%%%%%%%%%%%%%%%%%%%%%%%%%%%%
\documentclass[letterpaper]{article}
\usepackage{latexsym}
\usepackage{amsbsy}
\usepackage{graphicx}
\begin{document}
\noindent
\textbf{Use Case 0: Initialize Launch Data}\\
\textbf{Scope:  }Launch Simulator\\
\textbf{Level:  }Flight Controller Goal\\
\textbf{Primary Actor:  }Flight Controller\\
\textbf{Secondary Actor:  }None\\
\textbf{Related Use Cases:  }None\\
\textbf{Stakeholders \& Interrests:}
\begin{itemize}
   \item Flight Controller:  wants to initialize the Launch Sequence
      with the Appropriate Data
      \begin{itemize}
         \item Rocket 
         \item Launching Platform 
         \item Capsulue
      \end{itemize}
\end{itemize}
\textbf{Pre-Conditions:  }System is started:  ready to accept
initialization data\\
\textbf{Post-Conditions:  }Initializaiton data entered by the Flight
Controller--including Threshold data as needed\\\\
\textbf{Flow of Events: }\\
\begin{tabular}{|p{5.75cm}|p{5.75cm}|}\hline
\textbf{Flight Controller}&\textbf{System}\\\hline
1. Enters the initial Rocket Data  &\\\hline
2. For each Stage, enters the initial \& threshold Stage
Data & \\\hline
& 3. Saves the Rocket/Stage Data\\\hline
& 4. Displays Rocket/Stage Data\\\hline
5. Enters the Launching Platform Data&\\\hline
6. For each Launching Platform Support, enters the initial support
data&\\\hline
& 7. For each Support, calculates the Support Threashold Data based on
the initial Support input data \\\hline
& 8.  Saves the the Lauching Platorm Support \& Threshold Data\\\hline
&9. Saves the Input \& Calclulated Threshold Data for each
Support\\\hline
&10. Displays the Platform Support Data\\\hline
&11. Displays the Individual Support Data for every Support\\\hline
\end{tabular}\\\\
\textbf{Alternative Flows:}
\begin{itemize}
\item[*a.]If at any time, errant data is entered into the System,
then the System Alerts the Actors. The System will display
default data instead of errant data.  The Flight Director has the
choice to change the errant input data.
\item[*b.] If at any time, the System is unable to save the data, then
the System alerts the Actors to the Issue.
\end{itemize}
\textbf{Special Requirements: }
\begin{itemize}
\item None
\end{itemize}
\textbf{Technology \& Variations List: }
\begin{itemize}
\item[1c,2c,5c,6c] There are many ways to input the Initialized data.
How the Initialized data is entered determines the input protocol
and how errors are reported.
\item[1d,2d,5d,6d] Based on the components of the System (Rocket,
Lauching Mechanism, Capsule), the initialized input/threshold and
caculated threashold data will vary.
\end{itemize}
\textbf{Open Issues}
\begin{itemize}
\item To remove the ambiguity, consider using a Text file to
input the initialized and threshold data?  The System would parse the
text file for the initialized/threshold data
\end{itemize}
\end{document}
%%%%%%%%%%%%%%%%%%%%%%%%%%%%%%%%%%%%%%%%%%%%%%%%%%%%%%%%%%%%%%%%%%%%%%
