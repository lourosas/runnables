%%%%%%%%%%%%%%%%%%%%%%%%%%%%%%%%%%%%%%%%%%%%%%%%%%%%%%%%%%%%%%%%%%%%%%
\documentclass[letterpaper]{article}
\usepackage{latexsym}
\usepackage{graphicx}
\usepackage{amsmath}

\begin{document}
\title{The Lunar Mission Simulator Initial Writeup}
\author{Lou Rosas}
\maketitle

\section{History}
The great challenge of coming up with a real time Lunar Simulator.
This challenge was completed in the late 1960's via the Apollo Space
program.  Simulation of the entire mission (Lunar) with modern
computer hardware, and software design is the great challenge.

\section{Abstract}
With modern computation, a lunar simulation is now possible
for a typical home computer (or, even single board computer, like a
RaspberryPi).  This is the attempt:  to create such a simulation.

\section{Concept}
To gain understanding of developing a full simulation using
modern software design and techniques.  This includes leveraging
modern concurrency concepts and developments.

\section{Intent}
Develop a Lunar Simulator.  This is broken up into several separate
Simulations.
\begin{enumerate}
\item Launch Simulator
\item Orbital Simulator
\item Injection Simulator
\begin{enumerate}
\item Trans-Lunar Injection
\item Trans-Earth Injection
\end{enumerate}
\item Lunar Lander/Landing Simulator
\item Re-Entry Sequence
\end{enumerate}
Each one of these will need their own Initial Writeup.

\section{Stakeholders \& Interrests}
Anyone interrested in Using the Simulator.
\begin{itemize}
\item Space Crews--who want to simulate a complete Lunar Mission
\item Mission Planners--who want to ensure a complete and successful
mission.
\item Mission Planners--who want to simulate possible incidents
and/or accidents in that could happen durring a typical Lunar
Mission.
\item Engineers/Technicians--who want to view/assess mission data in
realtime in the Mission decision making process.
\end{itemize}

\section{Typical Success Scenario}
For a Lunar Mission, each sequence via each Simulator/Simulation is
complete with out indident.
\end{document}
%%%%%%%%%%%%%%%%%%%%%%%%%%%%%%%%%%%%%%%%%%%%%%%%%%%%%%%%%%%%%%%%%%%%%%
