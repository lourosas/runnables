%%%%%%%%%%%%%%%%%%%%%%%%%%%%%%%%%%%%%%%%%%%%%%%%%%%%%%%%%%%%%%%%%%%%%%
\documentclass[letterpaper]{article}
\usepackage{latexsym}
\usepackage{amsbsy}
\usepackage{graphicx}
\begin{document}
\noindent
\textbf{Use Case 1:  Play Game--NEEDS TO CHANGE!!!!!!!!!!!!!!}\\
\textbf{Scope:  }Sudoku Solver\\
\textbf{Level:  }Player Goal\\
\textbf{Primary Actor:  }Player\\
\textbf{Scenario:  }N/A\\
\textbf{Related Use Cases:  }\textit{The Player Shall Start a
New Game}\\
\textbf{Stakeholders \& Interests:}
\begin{itemize}
\item  Player:  Wants a solution to the puzzle
\end{itemize}
\textbf{Preconditions:  }Game is started\\
\textbf{Postconditions:  }Sudoku Puzzle completely solved\\
\textbf{Success Guarantees:  }A Solution to the initial Sudoku Puzzle
found and displayed to the Player\\\\
\textbf{Main Success Scenario:  }\\
\begin{tabular}{|p{5.75cm}|p{5.75cm}|}\hline
\textbf{Player} & \textbf{Sudoku Solver}\\\hline
1.  Play A Game (Solve a new puzzle) & \\\hline
& 2.  Request Player to enter a new puzzle\\\hline
3.  Enters a new puzzle via:
\begin{enumerate}
\item Text File
\item Manual Entry (Player Input)
\end{enumerate} & \\\hline
& 4.  Displays Unsolved Puzzle\\\hline
5.  Play the Game (Solve the Puzzle) &\\\hline
& 6. Solve the Puzzle\\\hline
& 7. Displays the ``evolving" solution\\\hline
& 8. Stops when solution completely solved\\\hline
& 9. Display completed solution\\\hline
\end{tabular}\\\\
\textbf{Extensions (Alternative Flows):}
\begin{itemize}
\item[3.1a.]If the text file cannot be opened then the System
indicates that to the Player, giving the player the choice to:
\begin{enumerate}
\item Open a different unsolved puzzle in a different text file
\item Perform 3.2 and Manually enter an unsolved puzzle
\end{enumerate}
\item[6a.]If there is no solution to the puzzle, then the System
alerts the Player
\end{itemize}
\textbf{Special Requirements:}
\begin{itemize}
\item None
\end{itemize}
\textbf{Technology \& Variatios List:}
\begin{itemize}
\item There are several different Sudoku Solvers available, future
incarnations of the Sudoku Solver plan to employ the different type
Solvers.
\end{itemize}
\end{document}
%%%%%%%%%%%%%%%%%%%%%%%%%%%%%%%%%%%%%%%%%%%%%%%%%%%%%%%%%%%%%%%%%%%%%%
