%%%%%%%%%%%%%%%%%%%%%%%%%%%%%%%%%%%%%%%%%%%%%%%%%%%%%%%%%%%%%%%%%%%%%%
\documentclass[letterpaper]{article}
\usepackage{latexsym}
\usepackage{graphicx}
\usepackage{amsmath}

\begin{document}
\title{The Sudoku Solver Initial Write-Up}
\author{Lou Rosas}
\maketitle

\section{History}
In my continued pursuit of concurrency understanding, I continue to
develop applications to leverage the process.  I checked on
programming sites on the web that concurrency would be required.  A
Sudoku Solver was suggested as one such application.

\section{Abstract}
A Sudoku Solver Application is an intriguing problem.  It is an
application that I have always wanted to make.  There are two types of
solutions that can be approached:  the ``Brute-Force” and the 
``Deductive Reasoning Type Algorithm”.

\section{Concept}
Create a Sudoku Solver Application.  That uses a ``Brute-Force”
Algorithm to find a solution to the Puzzle.  The puzzle is set either
with direct User input or via a text file.  The User should be able to
input the puzzle directly via some type of User entry.  If the User
chooses to input the puzzle via a text field, the application needs to
be able to put the values in the correct spots.  When reading in a
text file, any number less than 1 is to be considered an ``open spot”,
and shown as ``empty”, or no value in the Sudoku--this would be a
square that is solved.  The User shall have the ability to clear the
puzzle (same initial values, clear out the solution squares).  The
User shall have the ability to start a new game--a new puzzle with
different initial values in different locations in the Puzzle.  If a
solution is not attainable, the Application shall indicate that to the
User.

\section{Intent}
To create a Sudoku Solver as a way of practicing different ``Brute
Force” Algorithms and re-familiarizing recursion via the Sudoku 
``Engine”--different solution Engines for different ``Brute Force
Algorithms".

\section{Stakeholder \& Interests}
Sudoku Players who want to see a solution.
\end{document}
%%%%%%%%%%%%%%%%%%%%%%%%%%%%%%%%%%%%%%%%%%%%%%%%%%%%%%%%%%%%%%%%%%%%%%
