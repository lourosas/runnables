%%%%%%%%%%%%%%%%%%%%%%%%%%%%%%%%%%%%%%%%%%%%%%%%%%%%%%%%%%%%%%%%%%%%%%
\documentclass[letterpaper]{article}
\usepackage{latexsym}
\usepackage{graphicx}
\usepackage{amsmath}

\begin{document}
\title{The Coffee Maker Initial Write-up}
\author{Lou Rosas}
\maketitle

\section{History}
Concurrency is a rather elusive idea.  A rather “catch-22” concept.
And, it appears nothing is written to fully communicate the paradigm.
Rather, the literature on the subject appears to be long winded essays
on what not to do.
\section{Abstract}
My understanding of concurrency is incomplete.
I must take greater steps and action in its investigation.
Unfortunately, I do not pick up abstract concepts by just reading
about them.  I must ``play" with ideas, I must experiment with ideas
to fully ``grasp" them.
\section{Concept}
To gain understanding in concurrency, the intent is to model a Coffee
Maker in software.  This will improve the understanding of concurrency
via the development of different ways to handle concurrency.  
Mastering different ideas related to concurrency is the imperative.
As always, design is independent of implementation (language).  The
main concepts to be mastered in this development are Synchronization
(via Mutexs/Locks/Synchronized keyword/Synchronized(object)/\\Thread
Waiting/Thread Notification).
\section{Intent}
To model a Coffee Maker so as to expand the understanding of
concurrency.
\section{Stakeholders \& Interests}
The Coffee Drinker.
\section{Typical Success Scenario}
The Coffee Drinker Drinks coffee by pulling the carafe and filling a
mug or cup.\\
The Coffee Drinker makes coffee by filling the reservoir with water,
adding coffee grounds to an unused coffee filter at where the water
drips through the coffee grounds into the carafe.
\end{document}
%%%%%%%%%%%%%%%%%%%%%%%%%%%%%%%%%%%%%%%%%%%%%%%%%%%%%%%%%%%%%%%%%%%%%%
